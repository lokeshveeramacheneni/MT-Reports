%!TEX root = ../report.tex

\begin{document}
    \chapter{Conclusions}
    Advancements in Deep learning made the Deep Neural Networks (DNN) ready to deploy in real world.
    Nevertheless, they still suffer from the problem of misdetecting unknown objects in the environment.
    This is a significant problem to deal with in safety-critical applications.
    This thesis focuses on finding the objects (Out-of-Distribution) different from the training set in the 3D semantic segmentation setting.
    In this thesis, we surveyed the publicly available 3D LiDAR datasets.
    A survey on the state-of-the-art models for 3D semantic segmentation was also made, and chose RandLA-Net as the network of interest.
    Deep Ensembles and Flipout are used to estimate uncertainty, especially the entropy. 
    
    We propose two benchmark datasets for OOD detection, one being Semantic3D-vs-S3DIS (Outdoor-vs-Indoor) and Semantic3D-vs-Semantic3D without colour (colour-vs-no colour).
    The objects in the first OOD benchmark (S3DIS) are structurally different from the training dataset (Semantic3D).
    In the case of the second OOD benchmark (Semantic3D without colour), the colour feature is removed but the point geometry remains.
    This thesis predicts a point in the point cloud as an OOD point or not based on the Maximum Softmax Probability (MSP) or entropy.
    We observe that the both the OOD datasets has higher entropy scores and lower probability scores and vice-versa in the case of the training dataset.
    
    The difference in MSP and entropy between ID and OOD is high in the case of S3DIS, as the environment is very different from the training environment.
    In the case of the Semantic3D without colour, the difference in MSP and entropy to the ID dataset is thin, but the spread of probability and entropy values are higher for OOD dataset.
    We calculated the ROC curve between all the points in ID and OOD datasets, and generated the AUROC score to evaluate the model's performance of OOD detection with Deep Ensembles, Flipout and Dropout for multiple ensembles and forward passes, respectively.
    
    Overall, the performance of OOD detection is better with Deep Ensembles than with Flipout or Dropout in both the OOD datasets.
    This result complies with the results from OOD detection for the 2D classification task in \cite{lakshminarayanan2016simple}.
    Overall, we can conclude that the OOD detection in the case of unknown objects is possible with MSP or entropy.
    Whereas in the case of an OOD object having a similar structure to an object in the training set but with a difference in other attributes such as colour, the detection of OOD is challenging as the RandLA-Net model hugely rely on point geometry.
    
    \section{Lessons Learned}
    The following are the lessons learned during the duration of the thesis
    \begin{enumerate}
        \item Training and evaluation of 3D DNNs are time-consuming and resource-intensive. Even a single forward pass on the test set takes around 4 hours.
        \item LiDAR datasets have huge memory requirements even for preprocessing or computing metrics.
        \item With this level of requirements, finding a proper prior for Flipout layers is hard. In turn, this makes the tuning of the Flipout-versioned network hard.
        \item Picking the suitable candidates for the OOD benchmark requires in depth analysis of datasets like structural similarities and colour spectrum.
        \item We observe few points in the training (ID) dataset are classified as OOD points. These points have a lower probability score. So to get a perfect classification, post-hoc methods like MSP or entropy are of little use.
    \end{enumerate}

    \section{Future Work}
    
    We found three major ways to extend this thesis. 
    First one is regarding the 3D semantic segmentation models.
    We discussed three types of models point-based, projection-based and graph-based models.
    OOD detection in this thesis is limited to point-based model (RandLA-Net), in future this can be exteneded to other two types of models.
    Also the surveyed point-based and projection-based models have high number of parameters and in future, there is a scope for a model with better performance along with lower parameters.

    Second way to extend this thesis is with the OOD datasets.
    We grouped datasets into three kinds static, sequential and synthetic.
    The benchmark dataset proposed in this thesis is limited to static dataset and in future, benchmark datasets can include datasets from other types.
    Finally, the OOD detection methods are limited to post-hoc threshold methods on MSP and entropy scores in this thesis.
    So this can be extended to other OOD methods like Mahalanobis distance-based OOD \cite{lee2018simple_mahalanobis} or MetaSeg \cite{MetaSeg}.
    %The following are the ways this study can be further extended.
    %\begin{enumerate}
    %    \item Since this study is performed on the point-based model, this can be extended to a projection based model like RangeNet++ \cite{Hu_2020_CVPR_Randla} or SalsaNext \cite{SalsaNext_2020}.
    %    \item From the survey, we observe that the current 3D model's performance is not like its counterparts in 2D semantic segmentation. There is still a need for better performing model with lower parameters.
    %    \item The OOD datasets proposed in this dataset are static datatype which has a higher point density. It would be interesting to see sequential data as an OOD candidate as they have a lower point density.
    %    \item We observed another potential candidate for the OOD dataset, and it is the Toronto3D dataset. The Toronto3D dataset is an outdoor dataset, but objects such as buildings and trees are more American than the European setting in Semantic3D. Also, the size of the Toronto3D dataset is small compared to other datasets.
    %    \item The adverse weather conditions to the training dataset can be applied as \cite{fogsim} proposed LiDAR fog injection module, and this can be another possible way to create an OOD dataset from the training set.
    %    \item This study can be extended to apply non post-hoc methods such as Mahalanobis distance-based OOD \cite{lee2018simple_mahalanobis} or MetaSeg \cite{MetaSeg}.
    %\end{enumerate}
\end{document}
